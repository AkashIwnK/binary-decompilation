\subsection{\ISA Instruction Set Architecture}

\ISA is the 64-bit extension of X86, a family of backward-compatible ISAs.
% x64 is a generic name for the 64-bit extensions to Intel's and AMD's 32-bit x86 instruction set architecture (ISA). 
%
We briefly explain some details of the architecture.

% Registers:
\ISA has the sixteen 64-bit general purpose registers (\reg{rax}--\reg{rdx}, \reg{rsi}, \reg{rdi}, \reg{rsp}, \reg{rbp}, \reg{r8}--\reg{r15}), and the two $64$-bit special registers (\reg{rip} and \reg{rflags}).
The lower $32$-, $16$- and $8$-bit portions of the register are referenced by the sub-register variants, e.g., \reg{eax}, \reg{ax}, and \reg{al} for \reg{rax}, respectively.
% \cmt{All these registers are used for storing integer values.}
The Haswell \ISA ISA additionally has sixteen 256-bit SIMD registers (\reg{ymm0}--\reg{ymm15}) along with the lower 128-bit sub-register variants (\reg{xmm0}--\reg{xmm15}).
% \cmt{, which are used for floating point and packed operations.}

The \reg{rflags} register stores the current state of the processor.
Specifically, for example, the status flags such as the carry flag (\s{cf}), the parity flag (\s{pf}), the adjust flag (\s{af}), the zero flag (\s{zf}), and the sign flag (\s{sf}) are stored in \reg{rflags}.
These status flags are set according to the result of arithmetic and logical instructions.
% the status flags used mostly in user-level \ISA programs, and updated by arithmetic-logical instructions according to the result of the operation.
Some of control transfer instructions are performed based on the values of these flags.

% Addressing modes: 
\ISA supports the addressing modes, expressions that calculate a memory address to be read or written to. The addressing modes are used as the source or the destination of instructions that access the memory.
The addressing mode expressions can be generalized as:
$\s{base} + \s{index} \times \s{stride} + \s{offset}$.
In the assembly code, for example, \instr{-4(\reg{rax}, \reg{rbx}, 8)} denotes the address mode ``$\reg{rax} + \reg{rbx} \times 8 - 4$''.

% \cmt{      
% \begin{lstlisting}[style=SIMPRULES]
%     D (%RB, %RI, S)
%         RB is register for base
%         RI is register for index (0 if empty)
%         D is displacement (0 if empty)
%         S is scale 1, 2, 4 or 8 (1 if empty)
%     Effective Address: Mem[%RB + D + S*%RI]
% \end{lstlisting}
% }   

% Instruction variants:
% If an instruction access memory, we call it memory instruction. Else, if the instruction takes constant, then it is called immediate instruction. Otherwise, it is a register instructions.
\ISA has three types of instructions depending on the types of their operands: register instructions (with only register operands), memory instructions (with address mode operands), and immediate instructions (with constant operands).
The same mnemonic can be used for the different types of instructions.
For example, the mnemonic \instr{add} can be used for the register instructions (e.g., \instr{addq \reg{rax}, \reg{rbx}}\footnote{Throughout the presentation we will be using the {AT\&T} assembly syntax~\cite{Syntax} where the destination operand comes after source operands.}), the memory instructions (e.g., \instr{addq -4(\reg{rax}), \reg{rbx}}), and the immediate instructions (e.g., \instr{addq \$1, \reg{rbx}}).
% Throughout this paper, we will count each type of instructions as a different instruction.
% In the presentation, we will count them as different insructions ( and call them  instructions variants). Also different opcodes in the Intel manual are counted separately towards total instruction count. 
% \cmt{Whereas, register instructions differing on register names, immediate instructions differing on  constant values and memory instructions   differing on  memory addresses will be counted once towards total instruction count. }



