\subsection{K Framework}\label{sec:KF}

\K~\cite{k-primer-2013-v32} \cmt{\url{http://kframework.org}} is a framework for
defining formal language semantics. Given a syntax and a semantics of a language, \K
generates a parser, an interpreter, as well as formal analysis tools such as
model checkers and deductive program verifiers, at no additional cost. Using
the interpreter, one can test their semantics immediately, which significantly
increases the efficiency of semantics developments. Furthermore, the formal
analysis tools facilitate formal reasoning about the given language semantics.
This helps both in terms of applicability of the semantics and in terms of
engineering the semantics itself.

We   refer the reader to~\cite{k-primer-2013-v32, rosu-serbanuta-2010-jlap} for
 details. In a nutshell, in \K, a language syntax is given using conventional
Backus-Naur Form (BNF). A language semantics is given as a parametric
transition system, specifically a set of reduction rules over configurations. A configuration is an algebraic representation of the
program code and state. Intuitively, it is a tuple whose elements
(called cells) are labeled and possibly nested. Each cell represents a
semantic component, such as the memory or the registers. A special cell, named \s{k}, contains a
list of computations to be executed. A computation is essentially
a program fragment, while the original program is flattened into a
sequence of computations. A rule describes a one-step transition
between configurations, giving semantics to language
constructs. Rules are modular; they mention only relevant cells that
are needed in each rule, making many rules far more concise and easy to read
than in some other formalisms.



% One of the most appealing aspects of K is its modularity. It is very rarely the
% case that one needs to touch existing rules in order to add a new feature to
% the language. This is achieved by structuring the configuration as nested cells
% and by requiring the language designer to mention only the cells that are
% needed in each rule, and only the needed portions of those cells. For example,
% the above rule only refers to the \s{k} and \s{regstate} cells, while
% the entire configuration contains many more cells (Figure
% ~\ref{fig:config}). This modularity makes for compact and human
% readable semantics, and also helps with the overall effectiveness of the
% semantics development. For example, even if new cells are later added to
% configuration, to support new features, the above rule does not change.

% \begin{figure}[t]
  \centering
  %
  \renewcommand{\dotCt}[1]{\scriptstyle\s{#1}}
  \newcommand{\regstate}{\scriptstyle\s{ID}_\s{regname} \;\mapsto\; \s{Value}}
  \newcommand{\argc}{\scriptstyle\s{Int}}
  \newcommand{\argv}{\scriptstyle\s{Address}} 
  \newcommand{\entrypoint}{\scriptstyle\s{Address}}
%  \newcommand{\labeladdress}{\scriptstyle\textit{ID}_\textit{Label Name} \;\mapsto\; \textit{Address}}
  \newcommand{\codemem}{\scriptstyle\textit{Address} \;\mapsto\; \textit{Instruction}}
  \newcommand{\datamem}{\scriptstyle\s{Address} \;\mapsto\; \s{Value}}
  \newcommand{\bssmem}{\scriptstyle\textit{Address} \;\mapsto\; \textit{Value}}
  \newcommand{\stackmem}{\scriptstyle\textit{Address} \;\mapsto\; \textit{Value}}
  \newcommand{\heapmem}{\scriptstyle\textit{Address} \;\mapsto\; \textit{Value}}
  %
$
\kall{T}{
  \begin{array}{@{}c@{}}
  \kall{k}{\dotCt{K}} \mathrel{}
  \mathrel{}
  \kall{regstate}{\regstate} \mathrel{}
  \mathrel{}
  \kall{memstate} {\datamem} \mathrel{}
  \mathrel{}
  \cmt{ 
  \kall{executionEnv}{ 
  \kall{commandineArgs}{
    \kall{argc}{\argc} \mathrel{}    
    \kall{argv}{\argv} \mathrel{}
  } \mathrel{}
  \kall{entryPoint}{\entrypoint} \mathrel{}
}}
%  \kall{labelAddress}{\labeladdress} \mathrel{}
  \end{array}
}
$
  %
  \caption{Program Configuration}
  \label{fig:config}
\end{figure}

