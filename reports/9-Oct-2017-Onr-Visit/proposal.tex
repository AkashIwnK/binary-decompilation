%\documentclass[xcolor=dvipsnames]{beamer}
\documentclass[mathserif,12pt,unknownkeysallowed]{beamer}

\mode<presentation>
{
  \usetheme{UIUC}
}

\newcommand{\cmt}[1]{}
\setbeamercovered{transparent=50}
\newcommand{\LIR}{{\tt LLVM IR}}
\newcommand{\MC}{{\tt Machine Code}}
\newcommand{\CFG}{{\tt CFG}}
\usepackage{listings}
\usepackage{amsmath}
\usepackage{etoolbox}
\hypersetup{
  colorlinks=true,
  urlcolor=blue
}

\title[allvm]{allvm - Binary Decompilation}
\author{Sandeep Dasgupta}
\institute[UIUC]{University of Illinois Urbana Champaign}
\date{\today}


%if u want to show off the sections only
\setcounter{tocdepth}{1}

\lstset{language=[ANSI]C}
\lstset{% general command to set parameter(s)
  basicstyle=\footnotesize\tt, % print whole listing small
    identifierstyle=, % nothing happens
    commentstyle=\color{red}, % white comments
    showstringspaces=false, % no special string spaces
    lineskip=1pt,
    captionpos=b,
    frame=single,
    breaklines=true
      %\insertauthor[width={3cm},center,respectlinebreaks]
}



% If you want to come back to the subsection in outline everytime:
%\AtBeginSubsection
%{
%  \begin{frame}<beamer>{Outline}
%    \tableofcontents[currentsection,currentsubsection]
%  \end{frame}
%}

\AtBeginSection[] {
\ifnumcomp{\value{section}}{=}{5}{}
  {
  \begin{frame}[t]
    \frametitle{Outline}
      \tableofcontents[sectionstyle=show/shaded,hideothersubsections]
  \end{frame}
  }
}

\begin{document}

\begin{frame}
\titlepage
\end{frame}


%%%%%%%%%%%%%%%%%%%%%%%%%%%%%%%%%%%%%%%%%%%%%%%%%%%%%%%%%%%%%%%%%%%%%%%%%%%%%%%%%%%%%
\cmt{
Code that is only
available in binary form is “lifted” to fully executable LLVM IR by an extension of McSema [40]. This tool
is still experimental, and part of our proposed research is to extract richer information from this IR

}

\section*{Preliminary Work}
\subsection*{Preliminary Work}
\frame
{
  \frametitle{\subsecname}
      \begin{itemize}
        \item Improving the IR extracted by McSema.
          \begin{itemize}
            \item \href{https://github.com/sdasgup3/binary-decompilation/tree/stack_variable_recovery/tools/allin}{Stack frame deconstruction}
            \item Stack Variable promotion
            \item Type extraction
            \begin{itemize}
              \item Tools \href{https://github.com/sdasgup3/binary-decompilation/tree/stack_variable_recovery/tools/augment_ida_type}{Augment IDA Type} \& \href{https://github.com/sdasgup3/dwarf-type-reader}{Dwarf Type Reader}
            \end{itemize}
          \end{itemize}
      \end{itemize}

      \begin{itemize}
        \item Improving McSema Applicability.
          \begin{itemize}
            \item \href{https://github.com/trailofbits/mcsema/commit/ff83ee93840d3ab48f6abaf1a64908bbb0c63163}{Vector instruction support}
            \item \href{https://github.com/trailofbits/mcsema/commit/2ec9ee2342bc0721b73281f6aa758012f6afa035}{Translating unknown instructions into inline assembly}
          \end{itemize}
      \end{itemize}
}


\section*{Current Work}
  \subsection*{Current Work}
  \frame
  {
    \frametitle{\subsecname}
    \begin{description}
      \item [Problem:] Semantic translation from extracted CFG (from binary) to LLVM IR can be buggy \& difficult to extend.
      \item [Approach:]
        \begin{itemize}
          \item Learn the semantics rules automatically using strata.
          \item Use K framework to define the learned semantics.
          \begin{itemize}
            \item Help in validation the translation (decompilation) of binary to LLVM IR.\footnote{ Missing pieces like 1. Operational semantics  of LLVM IR in K \& 2. Language independent program equivalence checker, KEQ, are already in progress by other team.}
          \end{itemize}
        \end{itemize}
    \end{description}

  }




\end{document}
